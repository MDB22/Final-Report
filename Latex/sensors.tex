\label{sec:sensing}
Figure \ref{fig:sensing} outlines the on-board sensing capabilities that are available to the aircraft. The sections below detail the use of each sensor during a mission, and are separated according to the different flight modes introduced in Section \ref{}. 

\begin{figure}[!ht]
	\centering
	\includegraphics[width=300pt]{\IMAGEPATH sensors}
	\caption{Onboard sensing capabilities for Prototype \#1}
	\label{fig:sensing}
\end{figure}

\subsection{All Flight Modes}
\subsubsection*{Raspberry Pi}
The Raspberry Pi will act as the aircraft's on-board computing platform, providing autonomy by giving flight commands to the flight controller, as well as the processing and intelligence for path planning, and object detection. It will also pull flight data from the PixHawk and other sensors, and generate detailed flight logs for later review.

\subsubsection*{PixHawk Flight Controller}
The PixHawk will control the aircraft's flight functionality, such as controlling motors and ailerons, and executing flight paths and commands from the Raspberry Pi.

\subsection{Vertical Take-Off and Landing}
\subsubsection*{Ultrasonic Module}
The ultrasonic module will be mounted underneath the aircraft. The GPS and altimeter will provide altitude measurements during fixed-wing flight; the ultrasonic will augment these by providing a more reliable and controllable height measurement during rotor-based flight, assisting with search and landing.

\subsection{Long Range Flight}
\subsubsection*{PixHawk Sensors}
It also has several in-built or plug-and-play sensors, including a 3-axis accelerometer, altimeter, compass, and GPS. The PixHawk will provide the aircraft's telemetry to the Raspberry Pi and the base station, which will be augmented by the additional sensors below.

\subsection{Finding Joe}
\subsubsection*{Webcam}
The webcam will be mounted beneath the nose of the aircraft. It will provide vision for the aircraft's obstacle avoidance manoeuvers, and will form the basis for identifying Joe using his hat and blue jeans.

\subsubsection*{LiDAR}
The LiDAR will be mounted in the nose of the aircraft. The LiDAR can only measure the range of objects directly in front of it, so it will be mounted on a dual servo system that allows it to sweep a hemisphere in front of the aircraft, as shown in Figure \ref{fig:lidar}. It will provide a 3D map of the environment in front of the aircraft, and will assist in path planning and obstacle avoidance.

\begin{figure}[!ht]
	\centering
	\includegraphics[width=100pt]{\IMAGEPATH lidar}
	\caption{LiDAR mounting}
	\label{fig:lidar}
\end{figure}