\subsection{Achievements}
This project set out to develop a prototype UAV with a novel hybrid flight system, capable of competing in the 2016 UAV Challenge - Medical Express. The primary objectives identified in Section \ref{sec:intro} were
\begin{enumerate}[label=\bfseries O\arabic*:] \itemsep-2pt
	\item Register for the 2016 UAV Challenge
	\item Development of a prototype UAV for future teams to build on
	\item Development of autonomous flight controls to achieve \textbf{R1}
	\item Development of a novel hybrid flight system, incorporating both Vertical Take-Off and Landing (VTOL) and Fixed-Wing flight modes, for future teams to build upon to achieve requirements \textbf{R2}, \textbf{R3} and \textbf{R4}
	\item Development of in-flight search and obstacle avoidance mechanisms to be able to achieve \textbf{R5}
\end{enumerate}

Through extensive design and development, \ID have successfully achieved objectives \textbf{O1-O4}, and developed a strong foundation for competing in the 2016 UAV Challenge. Registration for the UAV Challenge was completed in September with the submission of Deliverable \#1, a short technical overview of our proposed aircraft.\\

Testing on the Dragonfly prototype proved that the aircraft was capable of sustained rotor-based flight, successfully achieving flight maneuver \textbf{M1}. The UAV is also equipped with a novel transition system, allowing it to convert between fixed-wing and VTOL flight modes at will, and providing benefits not found on either a purely fixed-wing or rotor-based aircraft. With the addition of the transition system, the prototype has the capability to land in any open terrain without a runway, unlike a regular fixed-wing aircraft, and has the capability to fly long-range or high-endurance missions, unlike a rotor-based aircraft.\\

The UAV is also capable of autonomous flight maneuvers, using a Raspberry Pi to interface with sensors throughout the aircraft, to generate and send flight paths to the PixHawk flight controller. The autonomous flight controls were successfully tested in simulation and on a proxy aircraft, and are ready for testing on the Dragonfly prototype. The software framework that was developed is also readily extensible to add complex flight behaviour not found in commercial UAVs, such as search, obstacle avoidance, and object detection to achieve objective \textbf{O5}.\\

Team \ID have successfully developed a cost-competitive autonomous UAV with hybrid flight, sensing and intelligence capabilities not found on current off-the-shelf products. While the UAV was designed in order to compete in the 2016 UAV Challenge, the novel flight system is cutting-edge research for autonomous aircraft, with applications ranging from emergency response, to delivery and transport, to defence, and far beyond.

\subsection{Further Work}
\red{Path planning\\

Prototype 3\\

Remaining submissions for UAV Challenge\\

\red{Install sensors - can we do this post report?\\}

Electronics architecture - wiring, circuit boards\\}

\subsection{Recommendations}
\red{A recent addition to the flight controller market, the NAVIO+\cite{ref:navio}, is designed for UAVs with vision control applications, and may be better suited for our application.\\
 
Naming of the next prototype ``Mosquito''.\\

Fixed wing automation \& control\\

Tests on battery and prop combo life times\\

Cooling or ventilation system\\

In-flight communications (Rocket M5s)\\

Look at last years winning team

Second Pi\\
}



\color{red}
Confirm that the objectives stated in the Introduction have been met. If the objectives in the Scope of Works document have not been fully met, an argument is required as to why the outcomes do not correspond with those envisaged.

Opportunities for further work, identified through the activities of the current project but outside its scope, should be identified.

This section will summarise your team’s final response to the initial “question, problem or issue”.  A summary of the arguments associated with your outcomes will be provided so that the reader is aware of your reasoning.

Do not include any personal responses to the project (eg. “...we enjoyed working with Joe and learnt a lot from Jen...”).  Write this report as if you are a professional practitioner, representing a research organization or consulting design bureau.

You are encouraged offer details of successful task completion. “Success” can be interpreted in many ways, for example: 
•	Team CP-xxxx contributed “X” to the overall “Y” research program led by Professor “Z”
•	“The client mentor was satisfied with the alternative conceptual designs offered by team CP-xxxx”
•	“The leader of the research division of the collaborating organisation was impressed with the alternative experimental method proposed by team CP-xxxx”
•	“An extensive review of the scientific literature has been completed by team CP-xxxx”
•	“Commercially available solutions were identified and ranked against criteria developed in conjunction with the client”

You can report on the status of your contributions.  For example, within the collaborating “research laboratory”, “research group”, “research initiative”, or “client company”, the final proposals of team CP-xxxx:
•	have been implemented,
•	are under review for later implementation,
•	are awaiting detailed costing, or
•	have provided a range of novel alternative strategies for later consideration.

Do not “apologise”.  Focus only on the positive outcomes of your work.  As an example, it is likely that tasks identified in your Scope of Works but not completed would have required more resources than were available.  Identify important tasks not completed as opportunities for further work within the associated DME laboratory or client organization, and discuss why they are important.  Given the many tasks that you have likely completed, your team now have an excellent knowledge of the requirements of the tasks not completed – briefly outline your expectation of the resources (i.e. personnel expertise, equipment, facilities, finance) needed to complete important tasks.
\color{black}