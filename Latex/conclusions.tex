\color{red}
Confirm that the objectives stated in the Introduction have been met. If the objectives in the Scope of Works document have not been fully met, an argument is required as to why the outcomes do not correspond with those envisaged.

Opportunities for further work, identified through the activities of the current project but outside its scope, should be identified.

This section will summarise your team’s final response to the initial “question, problem or issue”.  A summary of the arguments associated with your outcomes will be provided so that the reader is aware of your reasoning.

Do not include any personal responses to the project (eg. “...we enjoyed working with Joe and learnt a lot from Jen...”).  Write this report as if you are a professional practitioner, representing a research organization or consulting design bureau.

You are encouraged offer details of successful task completion. “Success” can be interpreted in many ways, for example: 
•	Team CP-xxxx contributed “X” to the overall “Y” research program led by Professor “Z”
•	“The client mentor was satisfied with the alternative conceptual designs offered by team CP-xxxx”
•	“The leader of the research division of the collaborating organisation was impressed with the alternative experimental method proposed by team CP-xxxx”
•	“An extensive review of the scientific literature has been completed by team CP-xxxx”
•	“Commercially available solutions were identified and ranked against criteria developed in conjunction with the client”

You can report on the status of your contributions.  For example, within the collaborating “research laboratory”, “research group”, “research initiative”, or “client company”, the final proposals of team CP-xxxx:
•	have been implemented,
•	are under review for later implementation,
•	are awaiting detailed costing, or
•	have provided a range of novel alternative strategies for later consideration.

Do not “apologise”.  Focus only on the positive outcomes of your work.  As an example, it is likely that tasks identified in your Scope of Works but not completed would have required more resources than were available.  Identify important tasks not completed as opportunities for further work within the associated DME laboratory or client organization, and discuss why they are important.  Given the many tasks that you have likely completed, your team now have an excellent knowledge of the requirements of the tasks not completed – briefly outline your expectation of the resources (i.e. personnel expertise, equipment, facilities, finance) needed to complete important tasks.
\color{black}