In order to participate in the UAV Challenge, any aircraft must be autonomous; an aircraft may only interact with a human to be armed or disarmed at the beginning and end of the challenge.

\subsection{Breaking Down the Challenge}
\label{sec:flight}
The Medical Express mission can be broken down into three discrete flight maneuvers:
\begin{enumerate}[label=\bfseries M\arabic*:] \itemsep-2pt
	\item Vertical take-off to cruising height, and landing
	\item Fixed-wing flight, navigating through waypoints and keeping within GeoFence boundaries
	\item Aerial search
\end{enumerate}

Using these maneuvers, completing the mission can be described as completing the sequence of actions\\
\begin{tabular}{r l l}
	1. & Mission start after being armed & (\textbf{M1}) \\ 
	2. & Navigate to Joe's location & (\textbf{M2}) \\ 
	3. & Aerial search to identify Joe & (\textbf{M3}) \\ 
	4. & Land near Joe to collect blood & (\textbf{M1}) \\ 
	5. & Take-off after being re-armed & (\textbf{M1}) \\ 
	6. & Navigate back to base & (\textbf{M2}) \\ 
	7. & Land at base & (\textbf{M1}) \\ 
\end{tabular} 

\subsection{Development}
\red{Autonomy on Raspberry Pi}

\subsection{Testing}
\red{Software in the Loop}