This project set out to develop a prototype UAV with a novel hybrid flight system, capable of competing in the 2016 UAV Challenge - Medical Express. The primary objectives identified in Section \ref{sec:intro} were
\begin{enumerate}[label=\bfseries O\arabic*:] \itemsep-2pt
	\item Register for the 2016 UAV Challenge
	\item Development of a prototype UAV for future teams to build on
	\item Development of autonomous flight controls to achieve \textbf{R1}
	\item Development of a novel hybrid flight system, incorporating both Vertical Take-Off and Landing (VTOL) and Fixed-Wing flight modes, for future teams to build upon to achieve requirements \textbf{R2}, \textbf{R3} and \textbf{R4}
	\item Development of in-flight search and obstacle avoidance mechanisms to be able to achieve \textbf{R5}
\end{enumerate}

Through extensive design and development, \ID have successfully achieved objectives \textbf{O1-O4}, and developed a strong foundation for competing in the 2016 UAV Challenge. Registration for the UAV Challenge was completed in September with the submission of Deliverable \#1, a short technical overview of our proposed aircraft.\\

Testing on the Dragonfly prototype proved that the aircraft was capable of sustained rotor-based flight, successfully achieving flight maneuver \textbf{M1}. The UAV is also equipped with a novel transition system, allowing it to convert between fixed-wing and VTOL flight modes at will, and providing benefits not found on either a purely fixed-wing or rotor-based aircraft. With the addition of the transition system, the prototype has the capability to land in any open terrain without a runway, unlike a regular fixed-wing aircraft, and has the capability to fly long-range or high-endurance missions, unlike a rotor-based aircraft.\\

The UAV is also capable of autonomous flight maneuvers, using a Raspberry Pi to interface with sensors throughout the aircraft, to generate and send flight paths to the PixHawk flight controller. The autonomous flight controls were successfully tested in simulation and on a proxy aircraft, and are ready for testing on the Dragonfly prototype. The software framework that was developed is also readily extensible to add complex flight behaviour not found in commercial UAVs, such as search, obstacle avoidance, and object detection to achieve objective \textbf{O5}.\\

Team \ID have successfully developed a cost-competitive autonomous UAV with hybrid flight, sensing and intelligence capabilities not found on current off-the-shelf products. While the UAV was designed in order to compete in the 2016 UAV Challenge, the novel flight system is cutting-edge research for autonomous aircraft, with applications ranging from emergency response, to delivery and transport, to defence, and far beyond.