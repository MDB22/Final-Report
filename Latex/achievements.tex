\red{WHAT DOES OUR AIRCRAFT BRING TO THE RESEARCH COMMUNITY:
\begin{itemize}
	\item Low cost autonomous aircraft (compare to Sony, defence drones, DJI and Parrot)
	\item Hybrid aircraft with transition capability (novel flight capabilities)
	\item Long range AND long flight time in a single unit (best of both aircraft)
	\item Foundations for intelligent drone flight (not just ``follow me'' behaviour)
	\item Achieved \textbf{M1} flight, with the capability of \textbf{M2} with successful transition, and \textbf{M3} with the installation of sensors
	\item Uncountable applications beyond medical transport (list)
\end{itemize}}

This project set out to develop a prototype UAV with a novel hybrid flight system, capable of competing in the 2016 UAV Challenge - Medical Express. The primary objectives identified in Section \ref{sec:intro} were
\begin{enumerate}[label=\bfseries O\arabic*:] \itemsep-2pt
	\item Register for the 2016 UAV Challenge
	\item Development of a prototype UAV for future teams to build on
	\item Development of autonomous flight controls to achieve \textbf{R1}
	\item Development of a novel hybrid flight system, incorporating both Vertical Take-Off and Landing (VTOL) and Fixed-Wing flight modes, for future teams to build upon to achieve requirements \textbf{R2}, \textbf{R3} and \textbf{R4}
	\item Development of in-flight search and obstacle avoidance mechanisms to be able to achieve \textbf{R5}
\end{enumerate}

Through extensive design and development, \ID have successfully achieved objectives \textbf{O1-4}, and developed a strong foundation for competing in the 2016 UAV Challenge. Registration for the UAV Challenge was completed in September with the submission of Deliverable \#1, a short technical overview of our proposed aircraft. 