The 2016 UAV Development teams are being left with a strong foundation to build upon, but with a long way to go to win the UAV Challenge. \ID would like to leave future teams with the following recommendations:

\begin{itemize}
	\item The NAVIO+\cite{ref:navio}, a recent addition to the flight controller market, is designed for UAVs with vision control applications, and may be better suited for this application than the PixHawk
	\item A ventilation system was added to the prototype, but it is recommended that better alternatives be investigated for cooling the internal electronics
	\item Investigation of long range communication should begin with the communications system used by last year's winning team \cite{ref:2014winners}, which consisted of two Rocket M5 transmitters \cite{ref:rocket}
	\item The three ESCs are currently being used as the primary power source, however it is recommended that an alternative such as a battery elimination circuit (BEC) be implemented
	\item It is unclear if a single Raspberry Pi will be capable of supporting all the processing required during a mission, so the addition of a second Raspberry Pi may prove useful
	\item Dragonfly is only a step in the right direction, and it is recommended that teams develop a new prototype in 2016; this prototype needs to be named Mosquito, in honour of the Medical Express task
\end{itemize}

The current prototype has several capabilities not yet available to commercial UAVs, but is not yet ready for participation in the Challenge. Team \ID believe that with these recommendations, and the future work above, the 2016 team will be strong contenders in the 2016 UAV Challenge.