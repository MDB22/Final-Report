\color{red}
(alex)
\begin{itemize}
\item all the stuff done
\item tests to prove it worked. Gyroscope equations
\item Anything that broke as a result of not testing something, or anything that resulted in modification to Dragonfly, write it as if we performed a test, then made the change, rather than ``oh that broke, what happened?''
\end{itemize}
\color{black}
In order to complete the objectives, long range flight would be required. This is why a transition system was necessary to implement, allowing the VTOL drone to fly forwards, both faster and more efficiently. This will hopefully allow the drone to fly for the required 1 hour flight time needed in deliverable 2, and allow it to travel for over 60 km that is required for the task. 
\\\\
The plane was designed from the beginning to deal with transition. What was decided was that two propellers of the VTOL copter would rotate forwards, creating a forward thrust and generating lift on the wings.  Before implementation however, a few physical considerations needed to be thought through.
\\\\
Firstly, a combination of wings and front motors capable of flight were required.  For this the Skywalker X8 seemed to be a commonly used drone that worked well, and the motors chosen (Turnigy SK3-3542-800) were the same ones chosen by YouTube user VilnisFPV who achieved over 142km of flight with the Skywalker. To ensure that it would definitely not stall mid-flight, the motors were checked on e-calc \color{red}[PUT IN APPENDIX REFERENCE TO ECALC]\color{black}, and they had an estimated stall speed of 32km/h with this combination, and a cruising speed of 82km/h.  The performance was also compared against other motors. 
\\\\

Next, an analysis of the gyroscopic forces acting on the motors during transition was needed to be done to determine the configuration of rotating motors, the strength of the shaft in between the two motors, and the torque required to make the transition. The calculations \color{red}[PUT CALCULATIONS IN APPENDIX SOMEWHERE]\color{black} showed that gyroscopic forces would create a moment perpendicular to the direction of rotation based on the direction of the motor angular velocity. To counteract this, two propellers on the front spinning in opposite directions would create zero net moment, and therefore the drone would be able to remain straight after transition. This was already the plan as counter rotating propellers also create a zero net angular momentum from the front in VTOL mode, making control much easier.  From this, the total moment created by this gyroscopic motion was \color{red}[2M]\color{black} in the centre of the shaft, however for our rotation speed, this was less than the moment gravity exerted on the shaft while the motors were hovering \color{red}[9.81*G*d]\color{black}. Lastly, in order to cope with any other gyroscopic forces (such as holding the props steady while the plane rolled, pitched or yawed), a servo was chosen so that it could hold up gyroscopic forces of the two motors that had an angular velocity (precession velocity) equal to the servos maximum rotational speed. For the servos chosen this was 9kg/cm (0.88Nm). These calculations were verified by holding the drone down in tests and ensuring transition was possible at full motor speed.
\\\\
For implementation a 1:1 gear system attached to the servo and motor shafts was installed \color{red}[PHOTO OF SYSTEM]\color{black}. This allowed for full rotation of the servo, and maximum accuracy.  Through testing of PWM values, it was set up to rotate by 90 degrees as required. A new mode was created through the open source Pixhawk firmware called “FixedWing”, and possible transition from the VTOL modes was also set-up. This was created so that when a switch on the controller is turned on the front motors rotate forwards, the VTOL control systems turn off, and the fixed wing control systems take over. So far, a manual fixed wing mode has been created. It changes the controls from controlling the rotation of the multi-copter, to the control surfaces of the plane when entering this mode.  Eventually other modes would need to be created to complete the task, such as automated modes, or modes with basic stability control. 
\\\\
\color{red}
Information pertaining to future flight test
\\\\
(For the diary: In order to implement changes to the firmware were required.  This meant having  a complete understanding of the programming behind both the Pixhawk and the open source ardupilot project, and it also meant we contributed to this open source platform.)
\color{black}
