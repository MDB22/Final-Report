\red{\begin{itemize}
\item more about software?
\item information pertaining to future test flights

\end{itemize}}


\subsection{Why Transition?}
In order to complete the objectives, long range flight would be required. This is why a transition system was necessary to implement, allowing the VTOL drone to fly forwards, both faster and more efficiently. This will hopefully allow the drone to fly for the required 1 hour flight time needed in deliverable 2, and allow it to travel for over 60 km that is required for the task. 

\subsection{Forward flight}
The plane was designed from the beginning to deal with transition. It was decided that two propellers of the VTOL copter would rotate forwards, creating a forward thrust and generating lift on the wings.  
\\\\
Firstly, a combination of wings and front motors capable of forward flight were required.  As mentioned the Skywalker X8 was shown to work well with forward flight, and the motors chosen (Turnigy SK3-3542-800) were the same chosen by others online, some who have achieved over 142km of flight time with the Skywalker(link?). eCalc (see Appendix \ref{sec:ecalc}, Figure \ref{fig:fixed}) suggested that at 3.5kg the plane had an estimated stall speed of 32km/h and a cruising speed of 82km/h with this combination. Additional calculations (see \ref{sec:stall}) further verfied flight by ensuring a worst case stall speed at 4kg, with the Skywalkers minimum coefficient of lift for level flight at 0.5, of 45.5km/h. 

\subsection{Gyroscopic Forces}
An analysis of the gyroscopic forces acting on the motors during transition was performed to determine the required configuration of the motors, the strength of the shaft in between the two motors, and the torque required to make the transition. The calculations (see  \ref{sec:gyro}) showed that gyroscopic forces would create a moment perpendicular to the direction of rotation based on the direction of the motor angular velocity. To counteract this, two propellers on the front spinning in opposite directions would create zero net moment, and therefore the drone would be able to remain steady after transition. This was already the plan as counter rotating propellers also create a zero net angular momentum from the front in VTOL mode, making control much easier.\\

From this, the total moment created by this gyroscopic motion was 3.1Nm in the centre of the shaft, however for our rotation speed, this was less than the moment gravity exerted on the shaft while the motors were hovering, which was 9.156Nm. Lastly, the transition system would need to hold the propellers steady for any other gyroscopic forces. In particullar if the plane rolls too fast, it would create a moment in the front. Due to the counter-rotating front motors however, these forces again would be opposite and counteracting.\\

The front servo would therefore receive no counter moment in flight. An arbitrary servo was chosen at 8.65kg/cm (0.85Nm), through testing it was found to behave satisfactorily against frictional or other unknown forces. This was verified by holding the drone down in tests and ensuring transition was possible at full motor speed.

\subsection{System Design}
For implementation a 1:1 gear system attached to the servo and motor shafts was installed (see Figure \ref{fig:gearsys}). This allowed for full rotation of the servo, and maximum accuracy.  Through testing of PWM values, it was set up to rotate by 90 degrees as required. A new mode was created through the open source Pixhawk firmware called “FixedWing”, and possible transition from the VTOL modes was also set-up. This was created so that when a switch on the controller is turned on the front motors rotate forwards, the VTOL control systems turn off, and the fixed wing control systems take over.\\

So far, a manual fixed wing mode has been created. It turns off VTOL control systems and changes the user input from controlling the rotation of the multi-copter, to the control surfaces of the plane when entering this mode. In order to implement this changes to the firmware were required.  This meant having  a complete understanding of the programming behind both the Pixhawk and the open source ardupilot project, and it also meant contributions were made to this open source platform. Eventually other modes would need to be created to complete the task, such as automated modes, or modes with basic stability control. \\
\todomessage{more about software? dev site references?}

\begin{figure}[!h]
	\centering
	\includegraphics[width=300pt]{\IMAGEPATH /Prototype/gear_system}
	\caption{Front gear and mounting system}
	\label{fig:gearsys}
\end{figure}

\subsection{Testing}
A number of steps needed to be taken to transition into fixed wing mode: \todomessage{should we reference firefly instruction vid?}
\begin{itemize}
\item In altitude hold keep the throttle at 50\% and move the plane forwards to create some initial speed
\item At a stable but fast speed level the plane
\item Flick the transition switch 
\item Lower the throttle
\item Pull up and raise the throttle as required
		
\end{itemize}
\todomessage{On ground and in flight test details}