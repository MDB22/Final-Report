\label{sec:AppA}
\red{Detailed work completed by the project team not included in the main body (calculations, sketches, details of activities not suited to the main body, e.g. raw data from experiments).}

\subsection{Analysis of 2014 Aircraft}
\label{sec:lastYear}
Research suggests that a Y6 configuration, with 3 sets of 2 coaxial motors (2 pairs at the front, 1 at the rear), is the best setup for this task, as shown on the FireFLY6 in Figure 7 above. However, due to its weight and construction, this is not possible on the 2014 model. Instead, it will have to be fitted in quadrotor formation, with four motors and rotors attached under the fuselage (Figure 4). To hover the aircraft (22kg) with 4 (3.5kg) motors equipped with 10cm propellers, in air with density 1.168 kg/m3 would require\\

\begin{tabular}{r c l}
	$P$ & $=$ & $v_{air}*F_{thrust}$\\
	& $=$ & $\sqrt{F_{thrust}/(\rho_{air}*\pi*r_{prop}^2)}*F_{thrust}$\\
	& $=$ & $F_{thrust}^{3/2}/\sqrt{\rho_{air}*\pi*r_{prop}^2}$\\
	& $=$ & $(mg)^{3/2}/\sqrt{N_{motors}^3*\rho_{air}*\pi*r_{prop}^2}$\\
	& $=$ & $2500W$\\
\end{tabular}
\vspace{6pt}
	
This requirement is per motor, under ideal conditions (100 percent motor and rotor efficiency). A cost effective motor for this is the Turnigy RotoMax, which cost 90AUD on sale at HobbyKing, bringing the cost of motors alone to 360AUD. However, more significant than the cost is the 75A requirement per motor (combined current of 300A) along with a 10 cell (3.7V) battery.  The most cost effective battery at HobbyKing capable of this is the Turnigy Nano-Tech 4400mAh Lipo Pack at 81AUD. \color{red} With 4 motors at 75 Amps, this battery will last approximately $t_{batt}=36004400/475=53s$. \color{black}\\
	\todo[inline]{Where does the 36004400 come from?}
	
Assuming a single take-off/landing manoeuvre can take at most $t_{batt}$, we would require four batteries to complete the challenge, costing 324AUD in total, with total extra weight of 4.8kg. However, we would require even more power to complete the fixed-wing flight of up to 60km just to reach Joe.\\
	
In addition to cost, the modifications necessary to make such a plane would be significantly more difficult, as the supports would need to hold much more weight, and the plane itself is made of wood and not foam. \color{red} The design of the craft (as both a quadcopter and fixed wing aircraft) on the larger craft would have a significant amount of aerodynamic drag and be less efficient. \color{black}\\
	\todo[inline]{This sentence doesn't make sense}
	
Last year's project was designed for a very different challenge where VTOL was not required. It was also designed as a multi-platform craft, where the fixed wing was designed for manufacture, something we certainly don't plan on doing this year with the wing. With consideration to all of the above points, we believe that purchasing a foam model airframe for UAV development is the best course of action in for the \ID.

\subsection{Gyroscopic Effects}
Angular momentum of propellers: $H = I\omega$\\
Maximum motor speed $\omega = 800\times16.8\times2\pi/60 = 1407.45rad/s$\\
Prop Inertia $I = 1/12\times M(L^2+B^2) = 1.16\times10^{-4}kgm^2$ (Overestimate, as it assumes equally distributed mass, and constant width)\\\\

[diagram above] Using a small angle approximation $\Delta H = \omega_p \times\Delta t \times I \times \omega$\\ 
$\omega_p$ = Maximum servo rotation speed (procession) = $60^o/0.11s = 9.52rad/s$\\
$dH/dt = \omega_p \times I \times \omega = M = 1.55Nm$\\\\

- [diagram above] Counter-rotating propellers stop the drone from yawing (and possibly spinning out of control on transition), as the moments act in opposite directions. They also act to make gyroscopic forces on the servo negligible due to them counteracting each other in a twist motion when the drone rolls fast, instead of putting a moment on the servos.\\
- Max moment in the centre of the front shaft is $2M = 3.1Nm$. If this is less than the moment on the front bar at hover ($2/3$ $\times $ mass of drone $\times$ distance to prop) then the shaft should be capable of withstanding the gyroscopic effects.\\
For our set up this is the case as the gyroscopic moment is less than the weight of the plane on the front shaft $3.1Nm < 9.156Nm$. If it can withstand the weight, it should be able to withstand the gyroscopic moments easilly.

\subsection{eCalc Modelling}
\label{sec:ecalc}