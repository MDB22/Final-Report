\color{red}
\begin{itemize}
\item Basically PR1(prototype/printing/etc) (WES)
\item Extra stuff... legs, new gears, glued on parts and changed placements for wings(Alex))
\item testing:validation(how we knew we could fly), iterative design (designs changed through testing (like the back gears)), and problems (like the motor, power module and radio, their effect, and how they were resolved), link to diary (as there were significatn set backs)... 
\item \begin{itemize}
	\item when we started flying really low (wes) 
	\item machines and mechanisms we tested on to make sure it wouldn’t break (wes) 
	\item flight tests and tweaks in courtyard (wes) 
	\item major flight tests that worked well (july 3 and September 6) 
	\item Tuning 
	\item Data LOGS!!! 
	\item Major crashes and broken parts (Either don't include this, or frame it in a positive note. We can't have a section on failures.)
	\item final transition
\end{itemize}
\end{itemize}
\color{black}

\subsection{Configuration}
A Y-3 configuration was selected over a Y-6 configuration, as it would lead to a lighter and cheaper aircraft. However, the aircraft was designed to be capable of being 'upgraded' to a T-6 configuration if the need arose.\\
Three motors would provide lift in VTOL mode, the front two motors being mounted vertically utilizing a mounting system capable of being rotated forwards $90^{\circ}$ to facilitate the transition system and fixed-wing flight. The rear motor mount system was designed to be tilted laterally through the use of a servo-motor to enable the yaw mechanism of the aircraft.\\
In order to enable proper stable and robust mounting of all motors and mounting systems, custom prototyping using 3D printed parts were designed to fit within the X-8 flying frame. The major components of which are split up into the `Motor Mounts', `Front  Mounting System' and the `Back Mounting System'.\\
The Motor Mounts were designed to secure each motor to the mounting poles. 
The Front Mounting System was designed to incorporate the transition system, while also being stable enough to withstand disturbances
Back Mounting System needed to incorporate the yaw servo-motor system

mention optimal 3D print orientations were investigated to strengthen against most possible failure modes.

\subsection{3D Printing}

\subsection{Calibration}

\subsection{Testing}
