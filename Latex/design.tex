\color{red}
\begin{itemize}
\item Basically PR1(prototype/printing/etc) (WES)
\item Extra stuff... legs, new gears, glued on parts and changed placements for wings(Alex))
\item testing:validation(how we knew we could fly), iterative design (designs changed through testing (like the back gears)), and problems (like the motor, power module and radio, their effect, and how they were resolved), link to diary (as there were significatn set backs)... 
\item \begin{itemize}
	\item when we started flying really low (wes) 
	\item machines and mechanisms we tested on to make sure it wouldn’t break (wes) 
	\item flight tests and tweaks in courtyard (wes) 
	\item major flight tests that worked well (july 3 and September 6) 
	\item Tuning 
	\item Data LOGS!!! 
	\item Major crashes and broken parts (Either don't include this, or frame it in a positive note. We can't have a section on failures.)
	\item final transition
\end{itemize}
\end{itemize}
\color{black}

\subsection{2014 Design}
Reference the appendix 
 
\subsection{Configuration and Component Selection}
In order to begin design and development on critical subsystems, a pre-designed `off the shelf' airframe was desired; for this purpose a Skywalker X-8 Flying Wing was selected as it offered a large amount of under canopy storage room, 'execllent` glide performance, fast low power cruise speeds and was stated to be capable of carrying a maximum all-up-weight (AUW) of 3.5kg\\
To achieve the novel transition system sought after from the start of the project, the optimal motor configuration was investigated for the job. It was realized that a selection of the motors would need to be capable of being rotated forward too support the transition function and for such a purpose a `tri-copter' formation was selected.\\
A Y-3 configuration was selected over a Y-6 configuration, as it would lead to a lighter and cheaper aircraft. However, the aircraft was designed to be capable of being 'upgraded' to a Y-6 configuration if the need arose.\\
The three motors would provide lift in VTOL mode, the front two motors being mounted vertically utilizing a mounting system capable of being rotated forwards $90^{\circ}$ to facilitate the transition system and fixed-wing flight. The rear motor mount system was designed to be tilted laterally through the use of a servo-motor to enable the yaw mechanism of the aircraft.\\
In order to enable proper stable and robust mounting of all motors and mounting systems, custom prototyping using 3D printed parts and iterative design was utilized to fit within the X-8 frame. The major components of design were split up into the `Motor Mounts', the `Front  Mounting System' and the `Back Mounting System'.\\
The Motor Mounts were designed to secure a motor to a corresponding mounting pole, and while the basic design was straight forward, ancillary items were necessary in order to provide a. An assortment of nyloc-nuts, spring-washers and hose clamps were utilized in order to fasten 
The Front Mounting System was designed to incorporate the transition system, while also being stable enough to withstand disturbances associated with being placed at the front of the aircraft.
Back Mounting System needed to incorporate the yaw servo-motor system

mention optimal 3D print orientations were investigated to strengthen against most possible failure modes.

\subsection{Design Plan}
\color{red}
basically how we intend to meet every requirement for the challenge. Put in circuit diagrams here, information regarding long range network transmitters, and basic intro to sensors, automation and transition. Basically introducing what the overall plan looks like and what we have covered from that. Maybe do a chart or table or something
\color{black}
\subsection{3D Printing}

\subsection{Calibration}

\subsection{Testing}
