\red{ testing:validation(how we knew we could fly), iterative design (designs changed through testing (like the back gears)), and problems (like the motor, power module and radio, their effect, and how they were resolved), link to diary (as there were significatn set backs)... mention optimal 3D print orientations were investigated to strengthen against most possible failure modes.}

\subsection{Prototype 1 "Scorpion"}
In order to ensure that the design worked, small hover tests were conducted at university on a seperate base \red{[ADD PIC]}. Any problems encountered could be fixed before securing parts to the main air frame.

\subsection{Testing Mechanisms}
\subsubsection*{Thrust:} Thrust tests were performed on each selected propellor in order to get a quantitive ranking on their effectiveness. This was done by 
\subsubsection*{Balance Table:}
\subsubsection*{Rope:}
\subsubsection*{3D print orientation?:}
\subsubsection*{Anymore?:}

\subsection{Iterative Design}
Through testing, a number of problems with the aircraft have been tweaked and resolved

\subsubsection*{Vibrations:} Numerous vibration related problems and set-backs were encountered during testing and evaluation. An assortment of nyloc-nuts and spring-washers were utilized in order to fasten components together in a vibration resistant manner. Also each motor sits on rubber mounts.
\subsubsection*{Motor Mounts:} As previously mentioned there had been many iterations of motor mount design. This is due to the numerous problems that were resolved to do with strength, vibration and fastening of the mounts. The final design addresses all issues.
\subsubsection*{Gears:} Initially 1:2 gear ratios were chosen for both front and back gear systems. Through testing however, it became evident that the servos could power a 1:1 system which allowed for larger teeth and hence overall strenth.
\subsubsection*{Wires and Soldering:}  \red{(this one is very recent, talk about tests, failure and resolution)}
\subsubsection*{Radio Failure:} Problems were encountered with the initial remote control  transmitter leading to strange behaviour in flight. Range tests were conducted, and they are believed to be faulty. The back up transitter (HK-T6A V2) was chosen and resolved these problems.
\subsubsection*{Motor Failure:} Mid-flight motor failure was encountered causing complete aircraft failure with many parts requiring repair. It is unknown whether this was caused due to manufacturing error in the motors, or debris flying into it. Either way future flights have been conducted with little debris and grass that cannot reach the propellers. 
\subsubsection*{Power Module Failure:}  The initial 3DR Power Module received was faulty and a cheaper third party one was used for a few flights. Extremely unpredictable behaviour of the craft followed, and led to another major crash. After many sesnor, current and voltage tests it was determined to be the power module. A new 3DR Module was purchased, and no issues have ensued since.  
\subsubsection*{GPS and Compass:}  It became evident from flight logs that the compass of the GPS system was inaccurate and sporadic. This was due to the magnetic field created by the PWM signal from the ESCs to the motors. It was avoided by placing the compass outside of the drone shell.\\\\
\red{show some failure pics in for the above section, and link to diary or gant entries}